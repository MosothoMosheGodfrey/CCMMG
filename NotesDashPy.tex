\documentclass{article}
\usepackage{enumitem}



 
\usepackage[T1]{fontenc}
\usepackage[utf8]{inputenc}
\usepackage{charter}
\usepackage{environ}
\usepackage{tikz}
\usetikzlibrary{calc,matrix}

%% Code by Claudio:
%% https://tex.stackexchange.com/a/197447/221452
%% Uses code by Andrew:
%% http://tex.stackexchange.com/a/28452/13304
\makeatletter
\let\matamp=&
\catcode`\&=13
\def&{%
	\iftikz@is@matrix%
	\pgfmatrixnextcell%
	\else%
	\matamp%
	\fi%
}
\makeatother

\newcounter{lines}
\def\endlr{\stepcounter{lines}\\}

\newcounter{vtml}
\setcounter{vtml}{0}

\newif\ifvtimelinetitle
\newif\ifvtimebottomline

\tikzset{
	description/.style={column 2/.append style={#1}},
	timeline color/.store in=\vtmlcolor,
	timeline color=red!80!black,
	timeline color st/.style={fill=\vtmlcolor,draw=\vtmlcolor},
	use timeline header/.is if=vtimelinetitle,
	use timeline header=false,
	add bottom line/.is if=vtimebottomline,
	add bottom line=false,
	timeline title/.store in=\vtimelinetitle,
	timeline title={},
	line offset/.store in=\lineoffset,
	line offset=4pt,
}

\NewEnviron{vtimeline}[1][]{%
	\setcounter{lines}{1}%
	\stepcounter{vtml}%
	\begin{tikzpicture}[column 1/.style={anchor=east},
		column 2/.style={anchor=west},
		text depth=0pt,text height=1ex,
		row sep=1ex,
		column sep=1em,
		#1
		]
		\matrix(vtimeline\thevtml)[matrix of nodes]{\BODY};
		\pgfmathtruncatemacro\endmtx{\thelines-1}
		
		\path[timeline color st]
		($(vtimeline\thevtml-1-1.north east)!0.5!(vtimeline\thevtml-1-2.north west)$)--
		($(vtimeline\thevtml-\endmtx-1.south east)!0.5!(vtimeline\thevtml-\endmtx-2.south west)$);
		
		\foreach \x in {1,...,\endmtx}{
			\node[circle,timeline color st, inner sep=0.15pt, draw=white, thick]
			(vtimeline\thevtml-c-\x) at
			($(vtimeline\thevtml-\x-1.east)!0.5!(vtimeline\thevtml-\x-2.west)$){};
			\draw[timeline color st](vtimeline\thevtml-c-\x.west)--++(-3pt,0);
		}
		
		\ifvtimelinetitle%
		\draw[timeline color st]([yshift=\lineoffset]vtimeline\thevtml.north west)--
		([yshift=\lineoffset]vtimeline\thevtml.north east);
		
		\node[anchor=west,yshift=16pt,font=\large]
		at (vtimeline\thevtml-1-1.north west)
		{\textsc{Timeline \thevtml}: \textit{\vtimelinetitle}};
		\else%
		\relax%
		\fi%
		
		\ifvtimebottomline%
		\draw[timeline color st]([yshift=-\lineoffset]vtimeline\thevtml.south west)--
		([yshift=-\lineoffset]vtimeline\thevtml.south east);
		\else%
		\relax%
		\fi%
	\end{tikzpicture}
}

\begin{document}
	
	\section*{Inroduction}
	
	
	Hi Everyone, thank you for joining me today. Let me start by apologising for moving this session so many times, as you know its the "deliverables race" season, so I have to ask in advance that we keep this discussion short. Are you comfortable with that ?\\
	
	OK. This topic "Reverse engeneering useful information into data" its certainly not a new topic - There have been so many attempts over the years to standardise it into our day-to-day lives or in our businesses without success - Because there are so many complex things to consider for this to be a reality. But with the adancements of technology this is no longer avoidable especially where money is involved. 
	
	So lets look at our topic "Coverting information into data" it seems confusing right, because information and data should be related in principle, but they are not the same, and their relationship is ...for a lack of a better word I would say "not reversible", this is something that most scholar will tell you.  Now basically, Information is data, but what type of data? I would say the one that has been organized, processed, and given context or meaning. But then on the other side, when we talk about data, this is basically just raw material in a form of numbers or text or  images, for example from which information can be derived by processing and interpretation.\\ So to be specific I mean verbal information and with data I mean numerical data here I am basicaly talking about numbers. Is this clear enough ?
	
	Now, why are we even talking about this, this has so many implications in our 
	
	
	
	Funraising event -Money raised depends on what you say.
	
	If this is successfully implemented, what is the best way of measuring information and convert it into numerical data. OK, should we use percentages for general information? And what is the percentage range ? Should we use 0 to 100\%, and what will 0\% or 100\% mean ? OR should we rate using money to gauge our market gains or losses and how do we go back and forth between these measures ?
	
	
	On the face of it this looks like an easy problem, I mean lets look at the first national speech that president Ramaphosa gave - Just look at the reaction from the social media for example and use that as a benchmark, what I mean by that ...if you can design a metric that calculates those comments as a positive/negative reaction/impact which can be translated as whether the people suppotes what he said or not, then all the national address following that can in principle be comapred to that in percentages, for example 51\% suppotes what he said or not, or can we measure his address in millions of rands looking at the impact its going to have on the stock markets for example. 
	
	

	 	\begin{itemize}
	 		
	 	\item 	{\bf{Q.} Do you think these merchant devices are going to replace ATMs in future ? Now think about that, while you're still thinking, let me take you back. We once had a technology in a form of pay-phones. There was someone who disliked that technology and came up with an innovative idea of taking the same technology, i.e you can make calls and anyone can still call you, but instead that person made it portable and mobile -and now we have cellphpone. That idea got rid of those stationary pay-phones. Are we seeing the same thing happening here.} ====> Ans. Well yes, Contactless Payments are gaining popularity and this coupled with the introduction of digital currency, I do not see how physical cash withdrawals are still going to continue. ====> Ans. The other thing we can look at is the Demographic Shifts - the Changes in our population that's what I am trying to say. Is our population aging? I think statsSA came give us good statistics. Because, I mean the Older generations might prefer traditional banking methods, i.e going to the branch or ATMs, while younger generations might be more comfortable with digital platforms.
	 		
		\item 	{\bf{Q.} What other innovative idea do you think can be implemented on ATMs so that people can still have an edge to use them in future?} ====> Ans. If digital currency is introduced, what is going to happen to that infrastracture? I think if there are signs that bank cards are still being used, then instead of dispancing physical cash, then maybe they can be used for card replacements. If you loose your card over the weekend, you do not have to wait for Monday and go to the branch.
		 ====> Ans. I think maybe ATMs should have a dual interface, the standard ATM display and the mobile app display, because most people are used to their phone apps.
		 
		 \item 	{\bf{Q.} Do you I think maybe in future banks can agree to create ONE ATM for all banks, sort of a super ATM, this ATM will have an interface that shows all the banks, you just chose the one that you're banking with, you do your business and once you're done, you just press exit then it will go back.  }
		

	\end{itemize}
	
	
\clearpage
 
 Thank you for joining me today, I will update you soon about the next topic. Again, Thank you.
 
 
\end{document}

 



 
	
 










\begin{document}
	
	\begin{vtimeline}[description={text width=7cm},
		row sep=4ex,
		use timeline header,
		timeline title={The title}]
		1947 & AT and T Bell Labs develop the idea of cellular phones\endlr
		1968 & Xerox Palo Alto Research Centre envisage the `Dynabook'\endlr
		1971 & Busicom 'Handy-LE' Calculator\endlr
		1973 & First mobile handset invented by Martin Cooper\endlr
		1978 & Parker Bros. Merlin Computer Toy\endlr
		1981 & Osborne 1 Portable Computer\endlr
		1982 & Grid Compass 1100 Clamshell Laptop\endlr
		1983 & TRS-80 Model 100 Portable PC\endlr
		1984 & Psion Organiser Handheld Computer\endlr
		1991 & Psion Series 3 Minicomputer\endlr
	\end{vtimeline}
	
	\begin{vtimeline}[timeline color=cyan!80!blue, add bottom line, line offset=2pt]
		1947 & AT and T Bell Labs develop the idea of cellular phones\endlr
		1968 & Xerox Palo Alto Research Centre envisage the `Dynabook'\endlr
		1971 & Busicom 'Handy-LE' Calculator\endlr
		1973 & First mobile handset invented by Martin Cooper\endlr
		1978 & Parker Bros. Merlin Computer Toy\endlr
		1981 & Osborne 1 Portable Computer\endlr
		1982 & Grid Compass 1100 Clamshell Laptop\endlr
		1983 & TRS-80 Model 100 Portable PC\endlr
		1984 & Psion Organiser Handheld Computer\endlr
		1991 & Psion Series 3 Minicomputer\endlr
	\end{vtimeline}
	
\end{document}
